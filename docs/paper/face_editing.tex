\section{Application 2: Face image editing using a 3D face model}
Intelligent manipulation of human facial images, such as expression editing is a hot topic in computer graphics. Here we introduce a system where inputting an image of a human face, the users can manipulate by moving the handle defined on the face mesh.

\subsection{Pipeline}
The pipeline is as follows:
\begin{enumerate}
\item Fit a face model to the image.
\item Manipulate the face model using laplacian mesh editing techniques.
\item Warp the original image according to the deformation field induced by the deformation of the underlying face model using the as-rigid-as-possible techniques.
\end{enumerate}

\subsection{Laplacian Mesh Editing}
To enable easy manipulation of the face mesh, we use the techniques called laplacian mesh editing.
Laplacian operator measures the flatness of the mesh:
$$\Delta f(\mathbf{x}) = \lim_{|B(\mathbf{x})| \rightarrow 0} \frac{1}{|B(\mathbf{x}))|} \int_{B(\mathbf{x})} f(\mathbf{z}) \;d\mathbf{z} - f(\mathbf{x})$$
Where $B(\mathbf{x})$ is an infinitesimal region around $\mathbf{x}$.
We want to the difference of mesh before and after deformation be small.  
\begin{align*}
\int_\Omega ||\Delta(\mathbf{x} - \hat{\mathbf{x}})||^2 \; d\mathbf{A} & \approx \text{tr}( \mathbf{D}^\top \mathbf{L}^\top \mathbf{M}^{-\top} \mathbf{M} \mathbf{M}^{-1} \mathbf{L}\mathbf{D})\\
&= \text{tr}(  \mathbf{D}^\top \underbrace{\mathbf{L}^\top \mathbf{M}^{-1} \mathbf{L}}_{\Q} \mathbf{D})
\end{align*}
where $\mathbf{D}, \mathbf{L}, \mathbf{M}$ is the difference of mesh, laplacian and the mass matrix respectively. 
$$
\min_{\D_\text{u}}
\tr \left((\D_\text{u}^\top \ \D_\text{h}^\top)
\left(\begin{array}{cc}
\Q_\text{u,u} & \Q_\text{u,h} \\
\Q_\text{h,u} & \Q_\text{h,h} 
\end{array}\right)
\left(\begin{array}{c}
  \D_\text{u} \\
  \D_\text{h}
\end{array}
\right)\right)
$$
$$
\min_{\D_\text{u}}
\tr\left(\D_\text{u}^\top \Q_\text{u,u} \D_\text{u} +
2 \D_\text{u}^\top \Q_\text{u,h} \D_\text{h} + 
\underbrace{\D_\text{h}^\top \Q_\text{h,h}
\D_\text{h}}_\text{constant}\right)
$$
$$
\min_{\D_\text{u}} 
\tr\left(
\D_\text{u}^\top \Q_\text{u,u} \D_\text{u} +
2 \D_\text{u}^\top \Q_\text{u,h} \D_\text{h})
\right)
$$
Set the gradient to zero
$$2 \Q_\text{u,u} \D_\text{u} + 2 \Q_\text{u,h} \D_\text{h} = 0 \rightarrow \D_\text{u} = \Q_\text{u,u}^{-1} \Q_\text{u,h} \D_\text{h}$$
Minimization w.r.t to the unconstrained points gives us the solution.

\subsection{As-rigid-as-possible image manipulation}
We can perform the laplacian mesh editing to get the deformed 3D model. Now we want to manipulate the image which corresponds to the model deformation. To achieve this, we first construct a mesh on the image and project the vertices of the 3D model to guide the deformation of the image using the as-rigid-as-possible shape manipulation techniques\cite{igarashi2005rigid}.

\subsection{Results}
Here we show three results using the techniques described above.
\begin{figure}[H]
  \includegraphics[width=\textwidth]{./img/laplacian.pdf}
\end{figure}
These three results shows that the system can perform 3D aware face image manipulation without artifact.
