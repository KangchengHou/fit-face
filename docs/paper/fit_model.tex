
\section{Fitting a face model}
Given a photo, we want to find a facial model fitted to the image. The overall procedure is to find the 2D feature points of the image, and use these 2D feature points as the proxy of the original image to fit a 3D face model.

\subsection{Bilinear face model}
First we introduce how to generate a 3D face model. A face model can be thought to be combined using two attributes, i.e. identity and expression. Bilinear face model \cite{vlasic2005face} assumes any face model is generated from these two attributes. 
$$\mathbf{f} = \mathcal{M} \times \mathbf{w}_{\text{id}}^\top \times \mathbf{w}_{\text{expr}}^\top$$
To implement this, we find the Basel Face Model \cite{gerig2017morphable}. What this face model database provides to us is slightly different from the original bilinear face model. 
\begin{align*}
\mathbf{f} = \bar{\mathbf{f}} + \mathcal{M}_{\text{id}} \times \mathbf{w}_{\text{id}}^\top + \mathcal{M}_{\text{expr}}\times \mathbf{w}_{\text{expr}}^\top \\
\mathbf{w}_{\text{id}} \sim \mathcal{N}(0, \text{diag}(\sigma_{\text{id}}^{(1)}, \dots, \sigma_{\text{id}}^{(N_\text{id})}) \\ 
\mathbf{w}_{\text{expr}} \sim \mathcal{N}(0, \text{diag}(\sigma_{\text{expr}}^{(1)}, \dots, \sigma_{\text{expr}}^{(N_\text{expr})})
\end{align*}
It can be regarded as Taylor expansion approximation of the blinear model. Though slightly different, the empirical result looks good so we didn't bother to spend time to find some database that use the original bilinear face model.
So a face model is decided by two parameters, the identity vector $\mathbf{w}_{\text{id}}$ and the expression vector $\mathbf{w}_{\text{expr}}$. To project the face model to the 2D plane, under the assumption of weak perspective camera model, with scale $s$, rotation $\mathbf{R}$ and translation $\mathbf{t}$, we have 
$$\hat{\mathbf{f}} = s (\mathbf{R} \mathbf{f} + \mathbf{t})$$
Now we know the generating process of a 3D face model:
$$\hat{\mathbf{f}} = s (\mathbf{R} (\bar{\mathbf{f}} + \mathcal{M}_{\text{id}} \times \mathbf{w}_{\text{id}}^\top + \mathcal{M}_{\text{expr}}\times \mathbf{w}_{\text{expr}}^\top) + \mathbf{t})$$
To fit 3D face model, we first use state-of-the-art commercial software provided by face++\footnote{https://www.faceplusplus.com.cn/} to track the feature points $\{l_1, \dots, l_{N_{\text{lm}}}\}$. Then we solve the following optimization problem:
$$\min_{\mathbf{w}_{\text{id}}, \mathbf{w}_{\text{expr}}, s, \mathbf{R}, \mathbf{t}} \sum_{i=1}^{N_{\text{lm}}}||l_i - h_i||_2^2$$
where $\{h_1, \dots, h_{N_{\text{lm}}}\}$ is the corresponding points on 3d model.
This is a hard optimization problem which can't be solved by singly derive the derivative of the objective function and set the derivative to zero. However, it turns out we can solve this hard optimization problem by coordinate descent by alternating the optimizaion of $\color{red}{s, \mathbf{R}, \mathbf{t}}$ and $\color{blue}{\mathbf{w}_{\text{id}}, \mathbf{w}_{\text{expr}}}$.
$$\min_{\color{blue}{\mathbf{w}_{\text{id}}, \mathbf{w}_{\text{expr}}}, \color{red}{s, \mathbf{R}, \mathbf{t}}} \sum_{i=1}^{N_{\text{lm}}}||l_i - \color{red}{s}\color{black}(\color{red}\mathbf{R}\color{black}(\mathcal{M} \times [\color{blue}{\mathbf{w}_{\text{id}}, \mathbf{w}_{\text{expr}}} \color{black}]^\top)_i + \color{red}{\mathbf{t}} \color{black})||_2^2$$
It turns out we can calculate the derivative w.r.t $s, \mathbf{R}, \mathbf{t}$ or $\mathbf{w}_{\text{id}}, \mathbf{w}_{\text{expr}}$ and set them to $0$. Thus we have the following algorithm.


\begin{algorithm}[H]
\KwIn{facial landmarks $\{l_1, \dots, l_{N_{\text{lm}}}\}$ and PCA model}
\KwOut{shape coefficients $\mathbf{w}$ and camera parameters $s,\mathbf{R},\mathbf{t}$}
Set $\mathbf{w} = \mathbf{0}$\;
\Repeat{$\mathbf{w}$ converges}{
    Set $\mathbf{f} = \bar{\mathbf{f}} + \mathcal{M} \times \mathbf{w}^\top$\;
    Find the camera parameters $s, \mathbf{R}, \mathbf{t}$ using $\mathbf{f}$ and $\{l_1, \dots, l_{N_{\text{lm}}}\}$\;
    Project all vertices of $\mathbf{f}$ onto the image plane: $\hat{\mathbf{f}} = s (\mathbf{R} \mathbf{f} + \mathbf{t})$\;
    Find the convex hull of $\hat{\mathbf{f}}$ as $\text{hull}(\hat{\mathbf{f}})$\;
    For contour landmarks $l_i$, find the correspondence\;
    Solve $\mathbf{w}$\;
}   
\caption{Fit face model to a single image}
\end{algorithm}
Finding the optimal $\mathbf{w}_{\text{id}}, \mathbf{w}_{\text{expr}}$ with $s, \mathbf{R}, \mathbf{t}$ fixed is a simple regularized least square problem. Finding the optimal $s, \mathbf{R}, \mathbf{t}$ with $\mathbf{w}_{\text{id}}, \mathbf{w}_{\text{expr}}$ fixed is a classical problem in computational photography and can be solved by POSIT algorithm\cite{dementhon1995model}.

\section{Application 1: Expression Flow for 3D-Aware Face
Component Transfer}

\section{Joint fitting of the model}
In some scenerios where we want to fit models for the same person that appeared in multiple images at the same time. We will need to fit the same identity weights $\mathbf{w}^{\text{id}}$ and fit the expression weights $\mathbf{w}^{\text{expr}}$ for different images. We adapt the previous algorithm as follows.

\section{Introduction of the applications}
Now we can fit a face model for a single image or fit multiple face models of the same person in multiple images, now we can do some interesting things. In the following sections, we will introduce three applications related to face model fitting and show the versatility of face fitting results.

\section{Face Image Editing}
Do some introduction of face image editing.
\subsection{Lapalacian Mesh Editing}

\subsection{Face IK}

\subsection{Results}


