\documentclass{article}
\usepackage{graphicx}
\usepackage{float}
\usepackage{amsmath}
\usepackage{amssymb}
\usepackage{amsfonts}
\usepackage{hyperref}
\usepackage{xcolor}
\usepackage{todonotes}
\usepackage{subfig}
\usepackage{caption}
\usepackage[ruled]{algorithm2e}

\newcommand{\D}{\mathbf{D}}
\newcommand{\Q}{\mathbf{Q}}
\DeclareMathOperator{\tr}{tr}
\title{Face Image Fitting\footnote{This is the first time we try to write the course project report in English. Your understanding is much appreciated.}}
\author{Kangcheng Hou\footnote{kangchenghou@gmail.com} \and Shuqi Wang\footnote{shuqiwang.cn@gmail.com} \and Tianhao Wei \footnote{phi.wth@gmail.com}}
\date{\today}
 

\newcommand{\pic}[2] {
    \begin{figure}[H]
    \centering
    \includegraphics[width=.6\textwidth]{./img/#1}
    \caption{#2}
    \end{figure}
}

\DeclareMathOperator*{\argmax}{arg\,max}
\DeclareMathOperator*{\argmin}{arg\,min}

\begin{document}

\maketitle
\tableofcontents

\section{Introduction}
A large proportion of images people uploaded is facial images. Motivated by this, in this course project, we want to develop a software that ease the process of editing facial image. The core part of our course project is face fitting, i.e. given a image, find the corresponding 3D model of the image. We also build two applications based upon the face fitting module.

The first one is to substitute the mouth in one image to another. The second one is to provide accessible way to edit face.
\begin{figure}
    \centering
    \subfloat[]{\includegraphics[width=0.4\textwidth]{./img/app11}}
    \hfill
    \subfloat[]{\includegraphics[width=0.4\textwidth]{./img/app12}}
    \caption{One application of face fitting is to transfer the smile from face (a) to face (b)}
\end{figure}
\begin{figure}
    \centering
    \subfloat[]{\includegraphics[width=0.4\textwidth]{./img/app21}}
    \hfill
    \subfloat[]{\includegraphics[width=0.4\textwidth]{./img/app22}}
    \caption{Another application of face fitting is to easily manipulate the face image without much artifact.}
\end{figure}

\subsection{Structure of the report}
In this report, we will first introduce the core part of this project, i.e. to reconstruct the 3D face model from a single image. Then we will introduce pipelines of two applications and their underlying techniques. Lastly, we will discuss further possible improvements of this project.
     

\section{Fitting a face model}
Given a photo, we want to find a facial model fitted to the image. The overall procedure is to find the 2D feature points of the image, and use these 2D feature points as the proxy of the original image to fit a 3D face model.

\subsection{Bilinear face model}
First we introduce how to generate a 3D face model. A face model can be thought to be combined using two attributes, i.e. identity and expression. Bilinear face model \cite{vlasic2005face} assumes any face model is generated from these two attributes. 
$$\mathbf{f} = \mathcal{M} \times \mathbf{w}_{\text{id}}^\top \times \mathbf{w}_{\text{expr}}^\top$$
To implement this, we find the Basel Face Model \cite{gerig2017morphable}. What this face model database provides to us is slightly different from the original bilinear face model. 
\begin{align*}
\mathbf{f} = \bar{\mathbf{f}} + \mathcal{M}_{\text{id}} \times \mathbf{w}_{\text{id}}^\top + \mathcal{M}_{\text{expr}}\times \mathbf{w}_{\text{expr}}^\top \\
\mathbf{w}_{\text{id}} \sim \mathcal{N}(0, \text{diag}(\sigma_{\text{id}}^{(1)}, \dots, \sigma_{\text{id}}^{(N_\text{id})}) \\ 
\mathbf{w}_{\text{expr}} \sim \mathcal{N}(0, \text{diag}(\sigma_{\text{expr}}^{(1)}, \dots, \sigma_{\text{expr}}^{(N_\text{expr})})
\end{align*}
It can be regarded as Taylor expansion approximation of the blinear model. Though slightly different, the empirical result looks good so we didn't bother to spend time to find some database that use the original bilinear face model.
So a face model is decided by two parameters, the identity vector $\mathbf{w}_{\text{id}}$ and the expression vector $\mathbf{w}_{\text{expr}}$. To project the face model to the 2D plane, under the assumption of weak perspective camera model, with scale $s$, rotation $\mathbf{R}$ and translation $\mathbf{t}$, we have 
$$\hat{\mathbf{f}} = s (\mathbf{R} \mathbf{f} + \mathbf{t})$$
Now we know the generating process of a 3D face model:
$$\hat{\mathbf{f}} = s (\mathbf{R} (\bar{\mathbf{f}} + \mathcal{M}_{\text{id}} \times \mathbf{w}_{\text{id}}^\top + \mathcal{M}_{\text{expr}}\times \mathbf{w}_{\text{expr}}^\top) + \mathbf{t})$$
To fit 3D face model, we first use state-of-the-art commercial software provided by face++\footnote{https://www.faceplusplus.com.cn/} to track the feature points $\{l_1, \dots, l_{N_{\text{lm}}}\}$. Then we solve the following optimization problem:
$$\min_{\mathbf{w}_{\text{id}}, \mathbf{w}_{\text{expr}}, s, \mathbf{R}, \mathbf{t}} \sum_{i=1}^{N_{\text{lm}}}||l_i - h_i||_2^2$$
where $\{h_1, \dots, h_{N_{\text{lm}}}\}$ is the corresponding points on 3d model.
This is a hard optimization problem which can't be solved by singly derive the derivative of the objective function and set the derivative to zero. However, it turns out we can solve this hard optimization problem by coordinate descent by alternating the optimizaion of $\color{red}{s, \mathbf{R}, \mathbf{t}}$ and $\color{blue}{\mathbf{w}_{\text{id}}, \mathbf{w}_{\text{expr}}}$.
$$\min_{\color{blue}{\mathbf{w}_{\text{id}}, \mathbf{w}_{\text{expr}}}, \color{red}{s, \mathbf{R}, \mathbf{t}}} \sum_{i=1}^{N_{\text{lm}}}||l_i - \color{red}{s}\color{black}(\color{red}\mathbf{R}\color{black}(\mathcal{M} \times [\color{blue}{\mathbf{w}_{\text{id}}, \mathbf{w}_{\text{expr}}} \color{black}]^\top)_i + \color{red}{\mathbf{t}} \color{black})||_2^2$$
It turns out we can calculate the derivative w.r.t $s, \mathbf{R}, \mathbf{t}$ or $\mathbf{w}_{\text{id}}, \mathbf{w}_{\text{expr}}$ and set them to $0$. Thus we have the following algorithm.


\begin{algorithm}[H]
\KwIn{facial landmarks $\{l_1, \dots, l_{N_{\text{lm}}}\}$ and PCA model}
\KwOut{shape coefficients $\mathbf{w}$ and camera parameters $s,\mathbf{R},\mathbf{t}$}
Set $\mathbf{w} = \mathbf{0}$\;
\Repeat{$\mathbf{w}$ converges}{
    Set $\mathbf{f} = \bar{\mathbf{f}} + \mathcal{M} \times \mathbf{w}^\top$\;
    Find the camera parameters $s, \mathbf{R}, \mathbf{t}$ using $\mathbf{f}$ and $\{l_1, \dots, l_{N_{\text{lm}}}\}$\;
    Project all vertices of $\mathbf{f}$ onto the image plane: $\hat{\mathbf{f}} = s (\mathbf{R} \mathbf{f} + \mathbf{t})$\;
    Find the convex hull of $\hat{\mathbf{f}}$ as $\text{hull}(\hat{\mathbf{f}})$\;
    For contour landmarks $l_i$, find the correspondence\;
    Solve $\mathbf{w}$\;
}   
\caption{Fit face model to a single image}
\end{algorithm}
Finding the optimal $\mathbf{w}_{\text{id}}, \mathbf{w}_{\text{expr}}$ with $s, \mathbf{R}, \mathbf{t}$ fixed is a simple regularized least square problem. Finding the optimal $s, \mathbf{R}, \mathbf{t}$ with $\mathbf{w}_{\text{id}}, \mathbf{w}_{\text{expr}}$ fixed is a classical problem in computational photography and can be solved by POSIT algorithm\cite{dementhon1995model}.




\section{Application 1: Expression Flow for 3D-Aware Face
Component Transfer}
We now introduce one application of face fitting. The goal of this application is to transfer the mouth from one to another. We first introduce the pipeline of this project and then introduce each components in more detail.
\begin{figure}[H]
\includegraphics[width=\textwidth]{./img/expression-flow-pipeline}
\caption{Step 1: use the off-the-shelf face tracker to get the 2D face landmarks. \\
Step 2: Perform joint face fitting on the two face images. \\
Step 3: Perform image warping to align the two images for later image compositing. \\
Step 4: Generate prior map and perform graph cut to find the best seam. \\
Step 5: Sometimes there will be visible artifact on the border, we do one step of Poisson image edting on the composited image.}
\end{figure}

\subsection{Joint fitting of the model}
In some scenerios where we want to fit models for the same person that appeared in multiple images at the same time. We will need to fit the same identity weights $\mathbf{w}^{\text{id}}$ and fit the expression weights $\mathbf{w}^{\text{expr}}_1, \dots, \mathbf{w}^{\text{expr}}_N$ for different images $I_1, \dots, I_N$. With slight adjustment of the previous single image fitting algorithm, we can get the joint fitting algorithm.

\subsection{Finding the best seam}
We can formulate this into a graph cut problem. In a graph cut framework. We first specify the data cost 
$$C(p) = \alpha \exp(-\frac{D_s(p)}{\sigma_d}) + (1 - \alpha) \left(1 - \exp(-\frac{||\nabla S(p)||}{\sigma_s})\right)$$
where $D_s(p)$ is the spatial distance from $p$ to the nearest pixel selected by the user, $||\nabla S(p)||$ is the gradient magnitude at $p$, and $\sigma_d$, $\sigma_s$ and $\alpha$ are parameters controlling the shape and weight of each term. $L(p)$ is the label of $p$. The data penalty in the graph cuts formulation is then defined as $C_d(p, L(p)) = 1 - C(p)$ is $L(p) = 1$(inside the crop region), and $C_d(p, L(p)) = C(p)$ if $L(p) = 0$(outside the crop region). 

Then we specify the smooth cost in the "match gradient" formulation for setting the neighborhood penalty $C_i(p, q, L(p), L(q))$ as:
$$||\nabla S_{L(p)}(p) - \nabla S_{L(p)}(p)|| + ||\nabla S_{L(p)}(q) - \nabla S_{L(q)}(q)||$$
This formulation encourages the cut to happen at the place where the difference of the gradients is small.


\subsection{Poisson Image Editing}
Here we introduce poisson image editing as a technique to remove visual seams. Visual seams occur because of the color mismatch between the two images. In poisson image editing we fix the colors of the boundary (taken from the background image) and provides a vector field that defines the structure of the image to be copied (taken from the foreground). The result image is generated by minimizing the squared error terms between the gradient of the result image and the guidance vector field.

The basic idea is to preserve the laplacian of the image while keep the boundary fixed.
$$\int_\Omega ||\Delta(\mathbf{x}_{1} - \hat{\mathbf{x}})||^2 \; d\mathbf{A} \quad \text{subject to } \hat{\mathbf{x}}^{\text{boundary}} = \mathbf{x}^{\text{boundary}}_2$$
Where the $\hat{\mathbf{x}}$ is the pixel value of inner part to be determined, $\mathbf{x}_1$ is the foreground image, $\mathbf{x}_2$ is the background image. 


\subsection{Results}
\begin{figure}[H]
    \centering
    \subfloat[]{\includegraphics[width=0.33\textwidth]{./img/flow2/img1}}
    \hfill
    \subfloat[]{\includegraphics[width=0.33\textwidth]{./img/flow2/img2}}
    \hfill
    \subfloat[]{\includegraphics[width=0.33\textwidth]{./img/flow2/blend_img}}
\end{figure}

\begin{figure}[H]
    \centering
    \subfloat[]{\includegraphics[width=0.33\textwidth]{./img/flow3/img1}}
    \hfill
    \subfloat[]{\includegraphics[width=0.33\textwidth]{./img/flow3/img2}}
    \hfill
    \subfloat[]{\includegraphics[width=0.33\textwidth]{./img/flow3/blend_img}}
\end{figure}

Here for each triple of images, we use the first image as the background image, the second as foreground image. Using the techniques described above, we get the third image as the result.

\section{Application 2: Face image editing using a 3D face model}
Intelligent manipulation of human facial images, such as expression editing is a hot topic in computer graphics. Here we introduce a system where inputting an image of a human face, the users can manipulate by moving the handle defined on the face mesh.

\subsection{Pipeline}
The pipeline is as follows:
\begin{enumerate}
\item Fit a face model to the image.
\item Manipulate the face model using laplacian mesh editing techniques.
\item Warp the original image according to the deformation field induced by the deformation of the underlying face model using the as-rigid-as-possible techniques.
\end{enumerate}

\subsection{Laplacian Mesh Editing}
To enable easy manipulation of the face mesh, we use the techniques called laplacian mesh editing.
Laplacian operator measures the flatness of the mesh:
$$\Delta f(\mathbf{x}) = \lim_{|B(\mathbf{x})| \rightarrow 0} \frac{1}{|B(\mathbf{x}))|} \int_{B(\mathbf{x})} f(\mathbf{z}) \;d\mathbf{z} - f(\mathbf{x})$$
Where $B(\mathbf{x})$ is an infinitesimal region around $\mathbf{x}$.
We want to the difference of mesh before and after deformation be small.  
\begin{align*}
\int_\Omega ||\Delta(\mathbf{x} - \hat{\mathbf{x}})||^2 \; d\mathbf{A} & \approx \text{tr}( \mathbf{D}^\top \mathbf{L}^\top \mathbf{M}^{-\top} \mathbf{M} \mathbf{M}^{-1} \mathbf{L}\mathbf{D})\\
&= \text{tr}(  \mathbf{D}^\top \underbrace{\mathbf{L}^\top \mathbf{M}^{-1} \mathbf{L}}_{\Q} \mathbf{D})
\end{align*}
where $\mathbf{D}, \mathbf{L}, \mathbf{M}$ is the difference of mesh, laplacian and the mass matrix respectively. 
$$
\min_{\D_\text{u}}
\tr \left((\D_\text{u}^\top \ \D_\text{h}^\top)
\left(\begin{array}{cc}
\Q_\text{u,u} & \Q_\text{u,h} \\
\Q_\text{h,u} & \Q_\text{h,h} 
\end{array}\right)
\left(\begin{array}{c}
  \D_\text{u} \\
  \D_\text{h}
\end{array}
\right)\right)
$$
$$
\min_{\D_\text{u}}
\tr\left(\D_\text{u}^\top \Q_\text{u,u} \D_\text{u} +
2 \D_\text{u}^\top \Q_\text{u,h} \D_\text{h} + 
\underbrace{\D_\text{h}^\top \Q_\text{h,h}
\D_\text{h}}_\text{constant}\right)
$$
$$
\min_{\D_\text{u}} 
\tr\left(
\D_\text{u}^\top \Q_\text{u,u} \D_\text{u} +
2 \D_\text{u}^\top \Q_\text{u,h} \D_\text{h})
\right)
$$
Set the gradient to zero
$$2 \Q_\text{u,u} \D_\text{u} + 2 \Q_\text{u,h} \D_\text{h} = 0 \rightarrow \D_\text{u} = \Q_\text{u,u}^{-1} \Q_\text{u,h} \D_\text{h}$$
Minimization w.r.t to the unconstrained points gives us the solution.

\subsection{As-rigid-as-possible image manipulation}
We can perform the laplacian mesh editing to get the deformed 3D model. Now we want to manipulate the image which corresponds to the model deformation. To achieve this, we first construct a mesh on the image and project the vertices of the 3D model to guide the deformation of the image using the as-rigid-as-possible shape manipulation techniques\cite{igarashi2005rigid}.

\subsection{Results}
Here we show three results using the techniques described above.
\begin{figure}[H]
  \includegraphics[width=\textwidth]{./img/laplacian.pdf}
\end{figure}
These three results shows that the system can perform 3D aware face image manipulation without artifact.
 
\section{Summary}
In this project, we have implemented the following:
\begin{itemize}
\item Single/Joint Face fitting \cite{yang2011expression}
\item Laplacian Surface Editing \cite{sorkine2004laplacian}
\item Poisson Image Editing \cite{perez2003poisson}
\item Formulate our problem into graph cut\cite{yang2011expression} and solve it using GCO library \cite{boykov2004experimental}
\item ARAP shape manipulation \cite{igarashi2005rigid}
\end{itemize}
Most importantly, we integate the techniques above to make two interesting applications.
 

    
\bibliographystyle{apalike}
\bibliography{ref}
 
\end{document}